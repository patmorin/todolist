\documentclass[lotsofwhite]{patmorin}
\listfiles
\usepackage{amsthm,amsmath,graphicx,wrapfig}
\usepackage[noend]{algorithmic}
\usepackage{pat}


\newcommand{\eps}{\varepsilon}

\title{\MakeUppercase{Top-Down Skiplists}}
\author{Luis Barba, Rolf Fagerberg and Pat Morin}


\begin{document}
\begin{titlepage}
\maketitle

\begin{abstract}
  We describe todolists (top-down skiplists), a variant of skiplists (Pugh
  1990) that can execute a searches using at most $\log_{2-\eps}
  n + O(1)$ binary comparisons per search and that has amortized update
  time $O(\eps^{-1}\log n)$.  A variant of todolists, called woselists
  (working-set skiplists) can execute a search for any element $x$ using
  $\log_{2-\eps} w(x) + o(\log w(x))$ binary comparisons and has amortized
  search time $O(\eps^{-1}\log w(w))$.  Here, $w(x)$ is
  the ``working-set number'' of $x$.  All previously-known structures with
  the working-set property perform at least $4\log_2 w(x)$ comparisons.
  We show through experiments that, if implemented carefully, todolists
  are comparable to other common dictionary implementations in terms of
  update times and outperform them in terms of search times.
\end{abstract}

\end{titlepage}

\section{Introduction}

Comparison-based dictionaries supporting the three \emph{basic operations}
insert, delete and search represent \emph{the} classic data-structuring
problem in computer science.  Solutions that support each of these
operations $O(\log n)$ time have been known for more than 50 years
\cite{avl}.  Since then, many competing implementations of dictionaries
have been proposed \cite{X}.  Most major programming environments include
one or more $O(\log n)$ time dictionary data structures in their standard
library \cite{S}.

In short, comparison-based dictionaries are so important that any
new ideas or insights on them are worth exploring.  In this paper,
we introduce the todolist (\boldx{to}p-\boldx{do}wn skip\boldx{list}), a
data structure that is parameterized by a parameter $\eps\in(0,1)$,
and that can execute searches using at most $\lceil(1+\eps)\log_2
n\rceil$ binary comparisons per search and that has amortized update
time $O(\eps^{-1}\log n)$.


\section{TodoLists}
\seclabel{todolist}

A \emph{todolist} for the values $x_1<x_2<\cdots<x_n$ consists of a
nested sequence of $h+1$ sorted singly-linked lists, $L_0,\ldots,L_h$,
having the following properties:\footnote{Here and throughout, we use set
notations like $|\cdot|$, and $\subseteq$ on the lists $L_0,\ldots,L_h$,
with the obvious interpretations.}

\begin{enumerate}
\item $|L_0| \le 1$.
\item $L_i\subseteq L_{i+1}$ for each $i\in\{0,\ldots,h-1\}$.
\item For each $i\in\{1,\ldots,h\}$ and each pair $x,y$ of consecutive elements in $L_i$, at least one of $x$ or $y$ is in $L_{i-1}$.
\item $L_k$ contains $x_1,\ldots,x_n$.
\end{enumerate}

The value of $h$ is at least $\lceil \log_{2-\varepsilon} n\rceil$ and at most
$\lceil \log_{2-\varepsilon} n\rceil+1$ .

We will assume that each list $L_i$ contains a \emph{sentinel} node
at the head of each list. This sentinel does not contain any data.
We will also assume that, given a pointer to the node containing $x_j$
in $L_i$, it is possible to find, in constant time, the occurence of $x_j$
in $L_{i+1}$.  This can be achieved by maintaining an extra pointer or by
maintaing all occurences of $x_j$ in an array. (See \secref{experiments}
for a detailed description.)

\subsection{Searching}

Searching for a value, $x$, in a todolist is simple. In particular, we
can find the node, $u$, in $L_h$ that contains the largest value that
is less than $x$. If $L_h$ has no value less than $x$ than the search
finds the sentinel in $L_h$.  We call the node $u$ the \emph{predecessor}
of $x$ in $L_h$.

Starting at the sentinel in $L_0$, one comparison (with the at most one
element of $L_0$) is sufficient to determine the predecessor, $u_0$ of $x$
in $L_0$. (This follows from Property~1.)  Moving down to the occurrence
of $u_0$ in $L_1$, one additional comparison is sufficient to determine
the predecessor, $u_1$ of $x$ in $L_1$. (This follows from Property~3.)
In general, once we know the predecessor of $x$ in $L_i$ we can determine
the predecessor of $x$ in $L_{i+1}$ using one additional comparison. Thus,
the total number of comparisons needed to find the predecessor of $x$
in $L_k$ is only $h+1$.

\vspace{1ex}
\noindent{$\textsc{FindSuccessor}(x)$}
\begin{algorithmic}
  \STATE{$u_0\gets \mathtt{sentinel}_0$}
  \FOR{$i=0,\ldots,k$}
    \IF{$\mathrm{next}(u_i)\neq \mathbf{nil}$ and $\mathrm{key}(\mathrm{next}(u_i)) < x$}
      \STATE{$u_i\gets\mathrm{next}(u_i)$}
    \ENDIF
    \STATE{$u_{i+1}\gets\mathrm{down}(u_i)$}
  \ENDFOR
  \RETURN{$u_k$}
\end{algorithmic}

\subsection{Adding}

Adding a new element, $x$, to a todolist is done by searching for it
using the algorithm outlined above and then splicing $x$ into each of
the lists $L_0,\ldots,L_h$.  This splicing is easily done in constant
time per list, since the new nodes containing $x$ appear after the nodes
$u_0,\ldots,u_h$.  At this point, all of the Properties~2--4 are satisified,
but Property~1 may be violated since there may be two values in $L_0$.

If there are two values in $L_0$, then we restore Property~1 with the
following \emph{partial rebuiding} operation: We find the smallest index
$i$ such that $|L_i|\le (2-\eps)^i$; such an index always exists since
$n=|L_h|\le(2-\eps)^h$.  We then rebuild the lists $L_{0},\ldots,L_{i-1}$
in a bottom up fashion; $L_{i-1}$ gets every second element from $L_i$
(starting with the second), $L_{i-2}$ gets every second element from
$L_{i-1}$, and so on down to $L_0$.

Since we take every element from $L_i$ starting with the second element,
after rebuilding we obtain:
\[
   |L_{i-1}| = \lfloor |L_i|/2 \rfloor \le |L_i|/2
\]
and, repeating this reasoning for $L_{i-2}, L_{i-3},\ldots, L_0$, we see that, after rebuilding,
\[
   |L_{0}| \le |L_i|/2^i \le (2-\eps)^i/2^i < 1 \enspace .
\]
Thus, after this rebuilding, $|L_0|=0$, Property~1 is restored and the
rebuilding, by construction, produces lists satisfying Properties~2--4.

To study the amortized cost of adding an element, we
can use the potential method with the potential function
\[
    \Phi(L_0,\ldots,L_h)=C\sum_{i=0}^h|L_i| \enspace .
\]
Adding $x$ to each of $L_0,\ldots,L_h$ increases this potential by
$C(h+1)=O(C\log n)$.  Rebuilding, if it occurs, takes $O(|L_i|)=O((2-\eps)^i)$
time, but causes a change in potential of at least
\begin{align*}
     \Delta\Phi & = C\sum_{j=0}^i\left(|L_j|/2^{i-j} - (2-\eps)^j\right) \\
          & \le C\sum_{j=0}^{i-1}\left((2-\eps)^i/2^{i-j} - (2-\eps)^j\right) \\
          & \le C\left((2-\eps)^i - \sum_{j=0}^{i-1}(2-\eps)^j\right) \\
          & = C\left((2-\eps)^i - \frac{(2-\eps)^i-(2-\eps)}{1-\eps}\right) \\
          & < C\left((2-\eps)^i - (1+\eps)\left((2-\eps)^i-(2-\eps)\right)\right)
           & \text{(since $1/(1-\eps)>1+\eps$)} \\
          & = -C\eps(2-\eps)^i + O(C) \\
\end{align*}
Therefore, by setting $C=c/\eps$ units for a sufficiently large constant,
$c$, the decrease in potential is greater than the cost of rebuilding.
We conclude that the amortized cost of adding an element $x$ is $O(C\log
n)=O(\eps^{-1}\log n)$.

\subsection{Deleting}

Since we already have an efficient method of partial rebuilding, we
can use it for deletion as well. To delete an element $x$, we delete
it in the obvious way, by searching for it and then splicing it out
of the lists $L_i,\ldots,L_h$ in which it appears.  At this point,
Properties~1, 2, and 4 hold, but Property~3 may be violated in any
subset of the lists $L_i,\ldots,L_h$.  Luckily, all of these violations
can be fixed by taking $x$'s successor in $L_h$ and splicing it into
each of $L_0,\ldots,L_{h-1}$.\footnote{If $x$ has not successor in
$L_h$---because it is the largest value in the todolist---then deleting
$x$ will not introduce any violations of Property~3.}  Thus, the second
part of the deletion operation is like the second part of the insertion
operation.  Like the insertion operation, this may violate Property~1
and trigger a partial rebuilding operation.  The same analysis used to
study insertion shows that deletion has the same amortized running time
of $O(\eps^{-1}\log n)$.

\subsection{Tidying Up}

Aside from the partial rebuilding caused by insertions and deletions, there are also some global rebuilding operations that are sometimes triggered:
\begin{enumerate}
\item If an insertion causes $n$ to exceed $\lceil(2-\eps)^h\rceil$, then
we increment the value of $h$ to $h'=h+1$ and rebuild $L_0,\ldots,L_{h'}$
from scratch, starting by moving $L_h$ into $L_{h'}$ and then performing
a partial rebuilding operation on $L_{0},\ldots,L_{h'-1}$.
\item If an insertion or deletion causes $\sum_{i=1}^n |L_i|$ to exceed $cn$ for some threshold constant $c>2$, then we perform a partial rebuilding to rebuild $L_{0},\ldots,L_{h-1}$.
\item If a deletion causes $n$ to be less than $\lceil(2-\eps)^{h-2}\rceil$ then we decrement the value of $h$ to be $h'=h-1$, move $L_h$ to $L_{h'}$ and then perform a partial rebuilding operation on $L_{0},\ldots,L_{h'-1}$. 
\end{enumerate}

A standard amortization argument shows that the first and third type
of global rebuilding contribute only $O(1)$ to the amortized cost of
each insertion and deletion, respectively.  The same potential argument
function used to study insertion and deletion works to show that the
second type of global rebuilding constributes only $O(\log n)$ to the
amortized cost of each insertion or deletion.  (Note that this second
type of global rebuilding is only required to ensure that the size of
the data structure remains in $O(n)$.)

This completes the proof of our first theorem:

\begin{thm}\thmlabel{todolist}
For any $\eps >0$, a todolist supports the operations of inserting,
deleting, and searching using at most $\log_{2-\eps} n + O(1)$ comparisons
per operation.  Starting with an empty todolist and performing any
sequence of $N$ add and remove operations takes $O(\eps^{-1}N\log
N)$ time.
\end{thm}

Note that, as a theoretical result, \thmref{todolist} is neither new
nor optimal.  The bound on the numbers on the number of comparisons
and the running-time is matched, for example, by scapegoat trees
\cite{galperin.rivest:blah,andersson:blah}.  Indeed, it is not even
optimal: Fagerberg, Lai, and Someone describe variants of 1-2 trees that
support all operations in $O(\log n)$ time and perform at most $\log_2
n + 1$ comparisons per operation.

\thmref{todolist} does compare favourably, however, with some of the
more popular structures found in textbooks and in programming libraries,
including red-black trees ($2\log_2 n$ comparisons), treaps ($1.38\log_2
n$ comparisons), skiplists ($1.88\log_2 n$ comparisons), and splay tree
($3\log_2 n$ comparisons).\footnote{The constants listed here for the
binary search trees (red-black trees, treaps, scapegoat trees, and splay
trees) are for ternary comparisons, which have three possible outcomes:
$a<b$, $a>b$, or $a=b$. If only binary comparisons are available, then
these constants should be doubled.} In \secref{experiments}, we will see
that a careful implementation of todolists actually outperforms all of
these data structures in terms of real search times, though the update
times are often slower.


\section{WoseLists}
\seclabel{woselist}

Next, we present a new theoretical result that is achieved using a
variant of the todolist that we call a woselist.  First, though, we need
some definitions.  Let $a_1,\ldots,a_m$ be a sequence whose elements come
from the set $\{1,\ldots,n\}$.  We call such a sequence an \emph{access
sequence}. For any $x\in\{1,\ldots,n\}$, the \emph{last-occurrence},
$\ell_i(x)$, of $x$ at time $i$ is defined as
\[
   \ell_i(x)=\max\{j\in{1,\ldots,i-1}: a_{j} = x\} \enspace .
\]
Note that $\ell_i(x)$ is undefined if $x$ does not appear in
$a_1,\ldots,a_{i-1}$.  The \emph{working-set number}, $w_i(x)$, of $x$
at time $i$ is
\[
    w_i(x) = \begin{cases}
               n & \text{if $\ell_i(x)$ is undefined} \\
               |\{a_{\ell_i(x)},\ldots,a_{i-1}\}| & \text{otherwise.}
             \end{cases}
\]
In words, if we think of $i$ as the current time, then $w_i(x)$ is the
number of distinct values in the access sequence since the most recent
access to $x$.

In this section, we describe the woselist data structure, which stores
$\{1,\ldots,n\}$ and, for any access sequence $a_1,\ldots,a_m$, can
execute searches for $a_1,\ldots,a_m$ so that the search for $a_i$
performs at most $(1+o(1))\log_{2-\eps} w_i(a_i)$ comparisons and takes
$O(\eps^{-1}\log w_i(a_i))$ amortized time.

From this point onward we will drop the subscripts on $w_i$ and assume that $w(x)$ refers to the working set number of $x$ at the current point in time (given the sequence of searches performed so far).
The woselist is a special kind of todolist that weakens Property~1 and adds an additional Property~5:

\begin{enumerate}
\item $|L_0|\le \eps^{-1}+1$.
\setcounter{enumi}{4}
\item For each $i\in\{0,\ldots,h\}$, $L_i$ contains all values $x$ such that $w(x)\le (2-\eps)^i$.
\end{enumerate}

For keeping track of working set numbers, a woselist also stores a
doubly-linked list, $Q$, that contains the values $\{1,\ldots,n\}$
ordered by their current working set numbers.  The node that contains $x$
in this list is cross-linked (or associated in some other way) with the
appearances of $x$ in $L_0,\ldots,L_h$.

\subsection{Searching}
\seclabel{todolist-search}

Searching in a woselist is similar to a search in a todolist.
The main
difference is that Property~5 guarantees that the woselist will reach
a list, $L_i$, that contains $x$ for some $i\le\log_{2-\eps} w(x)$.
If ternary comparisons are available, then this is detected at the
first such index $i$.  If only binary comparisons are available, then
the search algorithm is modified slightly so that, at each list $L_i$
where $i$ is a perfect square, an extra comparison is done to test if
the successor of $x$ in $L_i$ contains the value $x$.  This modification
ensures that, if $x$ appears first in $L_i$, then it is found by the
time we reach the list $L_{i'}$ for
\[
     i'=i+\left\lceil 2\sqrt{i}\right\rceil + 1 = \log_{2-\eps} w(x) + O(\sqrt{\log w(x)}) \enspace .
\]

Once we find $x$ in some list $L_{i'}$, we move it to the front of $Q$;
this takes only constant time since the node containing $x$ in $L_{i'}$
is crosslinked with $x$'s occurrence in $Q$.  Next, we insert $x$ into
$L_0,\ldots,L_{i'-1}$.  As with insertion in a todolist, this takes only
constant for each list $L_j$, since we have already seen the predecessor
of $x$ in $L_j$ while searching for $x$.  
At this point, Properties 2--5 are ensured and the ordering of $Q$ is
correct.  

All that remains is to restore Property~1, which is now violated
since $L_0$ contains $x$, which has $w(x)=1$ and the value $y$ such
that $w(y)=2$.  Again, this is corrected using partial rebuilding,
but the rebuilding is somewhat more delicate.  We find the first index
$i$ such that $|L_i|\le (2-\eps/2)^i$.  Next, we traverse the first
$(2-\eps)^{i-1}$ nodes of $Q$ and label them with their position in
$Q$; the label at a node of $Q$ that contains the value $z$ is then at
most $w(z)$.  

At this point, we are ready to rebuild the lists $L_0,\ldots,L_{i-1}$. To
build $L_{j-1}$ we walk through $L_j$ and take any value whose label
(in $Q$) is defined and is at most $(2-\eps)^j$ as well as every
``second value'' as needed to ensure that Property~3 holds.  Finally,
once all the lists $L_0,\ldots,L_j$ are rebuilt, we walk through the first
$(2-\eps)^{i-1}$ nodes of $Q$ and remove their labels so that these labels
are not incorrectly used during subsequent partial rebuilding operations.

\subsection{Analysis}

This completes the description of the algorithm for searching for an
element $x$ in a woselist.  We have already argued that we find a node
in some list $L_i$ with $i\in \log_{2-\eps} w(x) + O(\sqrt{\log w(x)})$
and that this takes $O(\log w(x))$ time.  The number of comparison needed
to reach this stage is
\[
     \log_{2-\eps} w(x) + O(\eps^{-1} + \sqrt{\log w(x)}) \enspace .
\]
The $O(\eps^{-1})$ term is the cost of searching in $L_0$. The 
$O(\sqrt{\log w(x)})$ term covers one comparison at each of the lists
$L_{\lceil\log_{2-\eps} w(x)\rceil},\ldots,L_i$ as well as the extra
comparison performed in each of the lists $L_j$ where $j\in\{0,\ldots,i\}$
is a perfect square.

After finding $x$ in $L_i$, the algorithm then updates
$L_0,\ldots,L_{i-1}$ in such a way that Properties~2--5 are maintained.
All that remains is to show that Property~1 is restored by the partial
rebuilding operation and to study the amortized cost of this partial
rebuilding.  Both of these begin by studying the sizes of the lists
$L_0,\ldots,L_i$.

Let $n_i=|L_i|$. Then,
the number of elements that make it from $L_i$ into $L_{i-1}$ is 
\[  |L_{i-1}| \le (2-\eps)^{i-1} + n_i/2 \enspace , \]
and the number of elements that make it into $L_{i-2}$ is
\begin{align*}
   |L_{i-2}| & \le (2-\eps)^{i-2} + |L_{i-1}|/2 \\
     & \le (2-\eps)^{i-2} + (2-\eps)^{i-1}/2 + n_i/4 \enspace . 
\end{align*}
More generally, the number of elements that make it into $L_j$ for any $j\in \{0,\ldots,i-1\}$ is at most
\begin{align*}
    |L_j| & \le (2-\eps)^{j} \cdot \sum_{k=0}^{j-i}\left(\frac{2-\eps}{2}\right)^k  + n_i/2^{i-j} \\
       & \le (2-\eps)^j/\eps + n_j/2^{i-j} \enspace .
\end{align*}
In particular
\[
    |L_0| \le \eps^{-1} + n_i/2^i \le \eps^{-1} + 1\enspace .
\]
Therefore, Property~1 is satsified.

To study the amortized cost of searching for $x$, we use the same
potential function argument as in \secref{todolist}.  The change in the sizes of the lists is then 
\begin{align*}
  \Delta\Phi/C & \le \sum_{j=0}^{i-1}\left((2-\eps)^j/\eps + n_j/2^{i-j} - (2-\eps/2)^j\right) \\
  & \le O((2-\eps)^i/\eps) + n_i - \sum_{j=0}^{i-1}(2-\eps/2)^j \\
  & = O((2-\eps)^i/\eps) + n_i - \frac{(2-\eps/2)^i}{1-\eps/2} + O(1) \\
  & \le O((2-\eps)^i/\eps) + n_i - (1+\eps/2)((2-\eps/2)^i) \\
  & \le O((2-\eps)^i/\eps) + n_i - (1+\eps/2)n_i \\
  & \le O((2-\eps)^i/\eps) - (\eps/2)n_i \\
  & \in - \Omega(\eps n_i) & \text{(since $n_i \ge n_{i-1} \ge (2-\eps/2)^{i-1}$)}
\end{align*}
Since the cost of rebuilding $L_0,\ldots,L_{i-1}$ is proportional to $O(n_i)$, this implies that the amortized cost of accessing $x$ is $O(\epsilon^{-1}\log w(x))$.


\section{Experimental Results}
\seclabel{experimental}
\seclabel{experiments}

To test the performance of the todolist, we implemented
it and tested them against other comparison-based dictionaries that
are popular, either in practice (red-black trees) and/or in textbooks
(scapegoat trees, treaps, and skiplists).  The implementation
was done in C++ and all the code was written by the third author.
(The implementations of all but todolists were adapted from the third author's
textbook \cite{S}.)  The code used to generate all the test data in this
section is available for download at github \cite{}.\footnote{The final
version of this paper will provide a DOI that provides a permanent fixed
version of the source code that was used to generate all figures in the
final paper.}

\subsection{The Story}

As a first attempt, one might try to implement a todolist exactly
as described in \secref{todolist}, with each list $L_i$ being a
separate singly linked list in which each node has a down pointer to
the corresponding node in $L_{i+1}$.  However, past experience with
skiplists suggests (and preliminary experiments confirmed) that this is
neither space-efficient nor fast (see the class \texttt{LinkedTodoList}
in the source code).

A better implementation uses one structure for each data item, $x$,
and this structure includes an array of pointers.  If $x$ appears in
$L_{i},\ldots,L_h$, then this array has length $h-i+1$ so that it can
store the \texttt{next} pointers for the occurrence of $x$ in each of
these lists.  This is, by now the standard implementation of skiplists
and this trick is suggested already by Pugh in his original paper on
skiplists \cite{pXX}.

Our initial implementation did exactly this, and performed well-enough
to beat standard skiplists but was still bested by most forms of binary
search trees.  This was despite the fact that the code for searching was
dead-simple, and by decreasing $\eps$ we could reduce the height, $h$ (and
hence the number of comparisons) to less than was being performed by these
search trees.

\subsection{The Problem With Skiplists}

After some thought, the reasons for the disapointing performance
of searches in a todolist became apparent and has to do with the
fact that following pointers is considerably more expensive than
performing comparisons.  In a binary search tree, there is a one-to-one
correspondence between comparisons and the nodes that are on a search
path.  In a todolist (and a skiplist as well) the search path visits a
sequence of nodes (these are the values that the variable $u$ takes on
in the pseudocode of \secref{todolist-search}).  However, the search
algorithm also performs comparisons with other nodes that are not on
the search path.

In a skiplist and in a todolist, the search algorithm performs comparisons
of the form $\mathrm{key}(\mathrm{next}(u)) < x$.  Here the node $u$
is on the search path, but the node $\mathrm{next}(u)$ whose key is
being accessed may never appear on the search path.  In a skiplist,
for example, this implies that there is a node in every list $L_i$
that is not on the search path but that is inspected during a search.
This is bad for the performance of skiplists and means that, for example,
a search inspects $3\log_2 n$ nodes on average even though the length
of the search path and the number of comparisons is only $2\log_2 n$.

\subsection{The Solution}

Luckily, there is a fairly easy remedy, though it does use more space.
We implement the todolist so that each node $u$ in a list $L_i$ stores
a pointer $\mathrm{next}(u)$ that, as usual points to $u$'s successor
in $L_i$ and $u$ also stores a key $\mathrm{keynext}(u)$ that is the
key associated with the node $\mathrm{next}(u)$.  This means that
determining the next node to visit after $u$ can be done using the
key $\mathrm{keynext}(u)$ stored at node $u$ rather than having to
dereference $\mathrm{next}(u)$.

Now, this trick is used in conjunction with the usual trick of storing all
the $\mathrm{next}$ values and $\mathrm{keynext}$ values for a key $x$
in an array that is attached to a single structure that also contains
the key $x$.  This is illustrated in \figref{packed-in}.

With this modification, the todolist structure achieves faster search
times, even with fairly large values of $\eps$ than the tree-based
structures. Indeed, retrofitting this idea into the skiplist
implementation improves its performance considerably so that it
outperforms some (but not all) of the tree-based structures.

\subsection{The Data}

TODO: This may even be faster than binary search on a sorted array, we should test this.

\section{Conclusion}

Todolists, when implemented properly seem quite difficult to beat, at
least in terms of search time.  They perform $\log_{2-\eps} n$ comparisons
and roughly half the comparisons follow a pointer to a successor
and then use an index calculation to move down into the next list.
The other half the comparison simply perform a trivial index calculation
(incrementing a pointer) to move straight down into the next list.
Thus, a search in a todolist does only about $\frac{1}{2}\log_{2-\eps}
n$ pointer dereferencing operations.

On the other hand, todolists leave a lot to be desired in terms of
insertion and deletion time.  Like other structures that use partial
rebuilding, the extra work (beyond searching) done during an insertion
takes time $\Omega(\log n)$, so is non-negligible.  

Although it's nice that todolists do a number of comparisons that is
approaching optimality, their memory layout is at least as important to
their performance.  It would be interesting to know if the memory layout
and data duplication tricks used in our implementation of todolists could
be applied to tree-based dictionaries.  The obvious generalization of this
to binary search trees is to store the tree as a collection of left-paths
each stored in an array. Such a representation would save a pointer
dereferencing operation each time the search path went to a left-child.
It's not immediately clear, however, how such a representation could be
combined with a tree-balancing scheme.

\end{document}


